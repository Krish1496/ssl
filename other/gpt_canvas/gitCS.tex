\documentclass{article}
\usepackage[a4paper,margin=1in]{geometry}
\usepackage{listings}
\usepackage{xcolor}
\usepackage{hyperref}

\definecolor{codegray}{gray}{0.9}
\lstdefinestyle{mystyle}{
    backgroundcolor=\color{codegray},
    basicstyle=\ttfamily\footnotesize,
    breaklines=true,
    frame=single,
    keywordstyle=\color{blue},
    commentstyle=\color{gray},
    stringstyle=\color{red}
}
\lstset{style=mystyle}

\title{Git Cheat Sheet}
\author{}
\date{}

\begin{document}
\maketitle

\section{Configuration}
\begin{lstlisting}
# Set username and email
git config --global user.name "Your Name"
git config --global user.email "your.email@example.com"

# Check current configuration
git config --list

# Set default editor
git config --global core.editor "vim"

# Enable color output
git config --global color.ui auto
\end{lstlisting}

\section{Initializing and Cloning}
\begin{lstlisting}
# Initialize a new Git repository
git init

# Clone a repository
git clone <repo-url>

# Clone a repository with a different name
git clone <repo-url> <new-folder-name>
\end{lstlisting}

\section{Basic Commands}
\begin{lstlisting}
# Check status of repository
git status

# Add all changes to staging area
git add .

# Commit changes with message
git commit -m "Your commit message"
\end{lstlisting}

\section{Branching}
\begin{lstlisting}
# List branches
git branch

# Create and switch to a new branch
git checkout -b <branch-name>

# Delete a branch
git branch -d <branch-name>
\end{lstlisting}

\section{Merging and Rebasing}
\begin{lstlisting}
# Merge a branch into current branch
git merge <branch-name>

# Rebase onto another branch
git rebase <branch-name>
\end{lstlisting}

\section{Remote Repositories}
\begin{lstlisting}
# Add a remote repository
git remote add origin <repo-url>

# Push changes to remote repository
git push origin <branch-name>
\end{lstlisting}

\section{Undoing Changes}
\begin{lstlisting}
# Revert to a previous commit
git revert <commit-hash>

# Reset to a previous commit (hard reset)
git reset --hard <commit-hash>
\end{lstlisting}

\section{Stashing Changes}
\begin{lstlisting}
# Stash current changes
git stash

# Apply the most recent stash
git stash pop
\end{lstlisting}

\section{Viewing Logs and History}
\begin{lstlisting}
# Show commit history
git log

# Show commit history in one line
git log --oneline
\end{lstlisting}

\section{Tagging}
\begin{lstlisting}
# Create a lightweight tag
git tag <tag-name>

# Push a tag to remote
git push origin <tag-name>
\end{lstlisting}

\section{Working with Submodules}
\begin{lstlisting}
# Add a submodule
git submodule add <repo-url>
\end{lstlisting}

\section{Git Ignore}
Create a `.gitignore` file to ignore specific files:
\begin{lstlisting}
# Ignore all `.log` files
*.log

# Ignore a specific folder
/node_modules/
\end{lstlisting}

\section{Git Aliases}
\begin{lstlisting}
# Create an alias for a command
git config --global alias.st status
\end{lstlisting}

\section{Security \& Authentication}
\begin{lstlisting}
# Generate SSH key
git ssh-keygen -t rsa -b 4096 -C "your.email@example.com"
\end{lstlisting}

\section{Force Push \& Fix Mistakes}
\begin{lstlisting}
# Force push changes (use with caution)
git push --force
\end{lstlisting}

\end{document}

